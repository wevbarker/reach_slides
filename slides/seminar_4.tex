\documentclass{beamer}
 
\usepackage[utf8]{inputenc}
\usepackage{siunitx}
\usetheme{CambridgeUS}
\useinnertheme{circles}

\usefonttheme[onlymath]{serif} 
 
%Information to be included in the title page:

\title{Seminar 4}
\subtitle{Special relativity}

\author{Will Barker\inst{1}\inst{2}}
\institute{
  \inst{1}%
    Cavendish Laboratory\\
    University of Cambridge\\
  \inst{2}%
    Kavli Institute for Cosmology\\
    University of Cambridge\\
}
\date{}
\logo{%
  \makebox[0.95\paperwidth]{%
    \includegraphics[height=0.7cm,keepaspectratio]{CU.eps}%
    \hfill%
    \includegraphics[height=0.7cm,keepaspectratio]{logo.png}%
  }%
}
 
 
 
\begin{document}
 
\frame{\titlepage}
 
\begin{frame}
  \frametitle{What is special relativity?}
  \begin{itemize}
    \item<1-> Most importanty the \textbf{regime} $v\sim c$
    \item<2-> This regime tends to \textbf{break} classical laws of phyics:
      \begin{itemize}
	\item<3-> \textbf{quantum mechanics}\to\textbf{quantum field theory}
	\item<4-> \textbf{classical mechanics}\to\textbf{relativistic mechanics}
      \end{itemize}
    \item<5-> Three space dimensions and one time dimension are \textbf{aspects} of a connected whole known as \textbf{spacetime}
  \end{itemize}
\end{frame}

\begin{frame}
  \frametitle{Special vs general relativity}
  \begin{itemize}
    \item<1-> What is the difference between \textbf{special} and \textbf{general} relativity?
    \item<2-> \textbf{General relativity} is needed when there is enough \textbf{mass}, \textbf{energy}, \textbf{momentum} or \textbf{stress} (or when the \textbf{densities} of these are high enough) that there is some \textbf{gravity} involved
    \item<3-> What does that mean about the \textbf{shape} of spacetime?
      \begin{itemize}
	\item<4-> \textbf{Special relativity}\to\textbf{flat spacetime}
	\item<5-> \textbf{General relativity}\to\textbf{curved spacetime}
      \end{itemize}
  \end{itemize}
\end{frame}

\begin{frame}
  \frametitle{Special vs general relativity}
  \begin{itemize}
    \item<1-> The \textbf{three} dimensional space we all live in is known as \textbf{Euclidean} space
    \item<2-> \textbf{Euclidean} space has some properties you are all very \textbf{familiar} with, but which we need to describe \textbf{mathematically} in order to compare with \textbf{non-Euclidean} space (i.e. \textbf{four} dimensional spacetime):
      \begin{itemize}
	\item<3-> We can \textbf{rotate} the space
	\item<4-> We can define \textbf{distances} which don't change when we \textbf{rotate} the space
      \end{itemize}
  \end{itemize}
\end{frame}

\begin{frame}
  \frametitle{We want everything simple\ldots}
  \begin{itemize}
    \item<1-> We will only deal with \textbf{two} dimensions at a time, otherwise everything will become very complicated!
  \end{itemize}
\end{frame}

\begin{frame}
  \frametitle{Euclidean rotations}
  \begin{itemize}
    \item<1-> Start with \textbf{two} ordinary space dimensions and coordinates $x$ and $y$, what is a \textbf{vector} which describes a \textbf{position} in that space?
    \item<2-> Should have:
      \begin{align*}
	\mathbf{x}=
	\begin{bmatrix}
	  x\\
	  y
	\end{bmatrix}
      \end{align*}
  \end{itemize}
\end{frame}

\begin{frame}
  \frametitle{Euclidean rotations}
  \begin{itemize}
    \item<1-> Now we need to introduce \textbf{matrices}!
    \item<2-> Just a \textbf{table} of numbers:
      \begin{align*}
	\mathbf{R}=
	\begin{bmatrix}
	  a & b\\
	  c & d
	\end{bmatrix}
      \end{align*}
  \end{itemize}
\end{frame}

\begin{frame}
  \frametitle{Euclidean rotations}
  \begin{itemize}
    \item<1-> Now we \textbf{multiply} a \textbf{vector} by a \textbf{matrix}!
    \item<2-> Here is the \textbf{rule} for doing this:
      \begin{align*}
	\begin{bmatrix}
	  x'\\
	  y'
	\end{bmatrix}
	=
	\begin{bmatrix}
	  a & b\\
	  c & d
	\end{bmatrix}
	\begin{bmatrix}
	  x\\
	  y
	\end{bmatrix}
	=
	\begin{bmatrix}
	 ax+by\\
	 cx+dy
	\end{bmatrix}
      \end{align*}
    \item<3-> \textbf{WRITE THIS DOWN, YOU WILL NEED IT}
  \end{itemize}
\end{frame}

\begin{frame}
  \frametitle{Euclidean rotations}
  \begin{itemize}
    \item<1-> \textbf{Over to you}: find $x'$ and $y'$ if:
      \begin{align*}
	\mathbf{R}=
	\begin{bmatrix}
	  \hphantom{-}\cos(\theta) & \sin(\theta)\\
	  -\sin(\theta) & \cos(\theta)
	\end{bmatrix}
	, \quad
	\mathbf{x}=
	\begin{bmatrix}
	  1\\
	  0
	\end{bmatrix}
      \end{align*}
    \item<2-> Emre: $\theta=10^\circ$, Mason: $\theta=360^\circ$, Ali Goktug: $\theta=45^\circ$, Claudia: $\theta=70^\circ$, Federico: $\theta=180^\circ$, Beltran: $\theta=270^\circ$
  \end{itemize}
\end{frame}

\begin{frame}
  \frametitle{Euclidean rotations}
  \begin{itemize}
    \item<1-> \textbf{Still over to you}: Now find $\sqrt{x'^2+y'^2}$ for these new vectors!
  \end{itemize}
\end{frame}

\begin{frame}
  \frametitle{Euclidean rotations}
  \begin{itemize}
    \item<1-> We should have found the following \textbf{exciting} things:
      \begin{itemize}
	\item<2-> The matrix \textbf{R} just \textbf{rotates} the position in space through the \textbf{angle}
	\item<3-> This leaves the \textbf{distance} from the origin \textbf{unchanged}
      \end{itemize}
  \end{itemize}
\end{frame}

\begin{frame}
  \frametitle{Euclidean rotations}
  \begin{itemize}
    \item<1-> So we can say this about \textbf{Euclidean} space:
      \begin{itemize}
	\item<2-> \textbf{Rotations} are done with \textbf{trigonometric} functions $\sin(\theta)$ and $\cos(\theta)$
	\item<3-> \textbf{Distances} are done like this: $\sqrt{x^2+y^2+z^2}$
      \end{itemize}
  \end{itemize}
\end{frame}

\begin{frame}
  \frametitle{Galilean rotations}
  \begin{itemize}
    \item<1-> This was rather \textbf{boring} actually, it gets more interesting if we have one \textbf{space} dimension and one \textbf{time} dimension
    \item<2-> It would be nice if we could set up this \textbf{two} dimensional space with coordinates which have the same \textbf{units}
    \item<3-> Let's say \textbf{space} is just given by $x$, what would be a good coordinate for time?
    \item<4-> So our new \textbf{position vector} is:
      \begin{align*}
	\mathbf{x}=
	\begin{bmatrix}
	  ct\\
	  x
	\end{bmatrix}
      \end{align*}
  \end{itemize}
\end{frame}

\begin{frame}
  \frametitle{Galilean rotations}
  \begin{itemize}
    \item<1-> We've just started to put together the idea of \textbf{spacetime}!
    \item<2-> Some important points:
      \begin{itemize}
	\item In \textbf{space} positions are usually known as \textbf{points}
	\item In \textbf{spacetime} the positions are known as \textbf{events}
      \end{itemize}
    \item<3-> Why?
  \end{itemize}
\end{frame}

\begin{frame}
  \frametitle{Galilean rotations}
  \begin{itemize}
    \item<1-> Now things begin getting complicated\ldots
    \item<2-> Let's say we start moving along the $x$ direction at \textbf{constant} velocity $v$ (couldn't think of a simpler motion really\ldots)
    \item<3-> At $t=0$, we were at $x=0$
    \item<4-> What is our $x$ at general $t$?
  \end{itemize}
\end{frame}

\begin{frame}
  \frametitle{Galilean rotations}
  \begin{itemize}
    \item<1-> Now things begin getting complicated\ldots
    \item<2-> Let's say we start moving along the $x$ direction at \textbf{constant} velocity $v$ (couldn't think of a simpler motion really\ldots)
    \item<3-> At $t=0$, we were at $x=0$
    \item<4-> What is our $x$ at general $t$?
    \item<4-> What is our $t$ at general $t$ (trick question!)?
  \end{itemize}
\end{frame}

\begin{frame}
  \frametitle{Galilean rotations}
  \begin{itemize}
    \item<1-> With this in mind, how will we measure $t'$ and $x'$ of some \textbf{event}:
      \begin{align*}
	\mathbf{x}=
	\begin{bmatrix}
	  ct\\
	  x
	\end{bmatrix}
      \end{align*}
    \item<2-> Just using common sense, we should end up with these very simple formulae:
      \begin{gather*}
	t'=t,\\
	x'=x-vt
      \end{gather*}
  \end{itemize}
\end{frame}

\begin{frame}
  \frametitle{Galilean rotations}
  \begin{itemize}
    \item<1-> These are the formulae for \textbf{Galilean} transformations:
      \begin{gather*}
	t'=t,\\
	x'=x-vt
      \end{gather*}
    \item<2-> You all already \textbf{knew these}: they just say that if you \textbf{move}, the \textbf{time} you observe an event is \textbf{the same}, but the position of the event \textbf{moves toward you} (or away if $v\to -v$)
    \item<3-> Does anyone \textbf{disagree}?
    \item<4-> These formulae are \textbf{completely and utterly wrong}, but nobody noticed until 1905!
  \end{itemize}
\end{frame}

\begin{frame}
  \frametitle{Galilean rotations}
  \begin{itemize}
    \item<1-> Note that I used the word \textbf{transformations} for $t\to t'$ and $x\to x'$, but \textbf{rotations} for $x\to x'$ and $y\to y'$
    \item<2-> That is because the idea of \textbf{rotating space} makes \textbf{perfect sense}, but \textbf{rotating space and time} sounds like \textbf{nonsense}\ldots
    \item<3-> Let's do it anyway!
  \end{itemize}
\end{frame}

\begin{frame}
  \frametitle{Lorentz rotations}
  \begin{itemize}
    \item<1-> \textbf{Over to you again}: find $t'$ and $x'$ if:
      \begin{align*}
	\mathbf{R}=
	\begin{bmatrix}
	  \frac{1}{\sqrt{1-v^2/c^2}} & \frac{-v/c}{\sqrt{1-v^2/c^2}}\\
	  \frac{-v/c}{\sqrt{1-v^2/c^2}} & \frac{1}{\sqrt{1-v^2/c^2}}
	\end{bmatrix}
	, \quad
	\mathbf{x}=
	\begin{bmatrix}
	  ct\\
	  x
	\end{bmatrix}
	=
	\begin{bmatrix}
	  1\\
	  0
	\end{bmatrix}
      \end{align*}
    \item<2-> Emre: $v=c$, Mason: $v=2c$, Ali Goktug: $v=0$, Claudia: $v=0.5c$, Federico: $v=0.2c$, Beltran: $v=0.1c$
  \end{itemize}
\end{frame}

\begin{frame}
  \frametitle{Lorentz rotations}
  \begin{itemize}
    \item<1-> \textbf{Next task}: find $\sqrt{c^2t'^2-x'^2}$ for your vectors!
  \end{itemize}
\end{frame}

\begin{frame}
  \frametitle{Lorentz rotations}
  \begin{itemize}
    \item<1-> \textbf{Now find this}:
      \begin{equation*}
	\left( \frac{1}{\sqrt{1-v^2/c^2}} \right)^2-\left( \frac{-v/c}{\sqrt{1-v^2/c^2}} \right)^2
      \end{equation*}
    \item<2-> \textbf{Finally who can remember what this is} for any $\psi$:
      \begin{equation*}
	\cosh(\psi)^2-\sinh(\psi)^2
      \end{equation*}
    \item<3-> Claudia: $\psi$ is another Greek letter pronounced `psi' ;) 
  \end{itemize}
\end{frame}

\begin{frame}
  \frametitle{Lorentz rotations}
  \begin{itemize}
    \item<1-> So it turns out we can just write that horrible matrix in a simple form:
      \begin{align*}
	\mathbf{R}=
	\begin{bmatrix}
	  \frac{1}{\sqrt{1-v^2/c^2}} & \frac{-v/c}{\sqrt{1-v^2/c^2}}\\
	  \frac{-v/c}{\sqrt{1-v^2/c^2}} & \frac{1}{\sqrt{1-v^2/c^2}}
	\end{bmatrix}
	=
	\begin{bmatrix}
	  \cosh(\psi) & \sinh(\psi)\\
	  \sinh(\psi) & \cosh(\psi)
	\end{bmatrix}
	=\mathbf{\Lambda}
      \end{align*}
    \item<2-> Claudia: $\Lambda$ is uppercase Greek letter `lambda', lowercase is $\lambda$ ;) 
  \end{itemize}
\end{frame}

\begin{frame}
  \frametitle{Lorentz rotations}
  \begin{itemize}
    \item<1-> So unlike last time, the things we get now should actually be interesting:
      \begin{itemize}
	\item<2-> When we move, space and time \textbf{rotate}, but instead of \textbf{trigonometric} rotations with $\sin(\theta)$ and $\cos(\theta)$ for some angle $\theta$ we have \textbf{hyperbolic} rotations with \textbf{hyperbolic} functions $\sinh(\psi)$ and $\cosh(\psi)$ -- don't bother with what $\psi$ actually is, there is a formula for it in terms of $v/c$ (Mason, find $\psi(v/c)$)
	\item<3-> The \textbf{distance} in spacetime is just $\sqrt{c^2t^2-x^2-y^2-z^2}$
      \end{itemize}
  \end{itemize}
\end{frame}

\begin{frame}
  \frametitle{Lorentz rotations}
  \begin{itemize}
    \item<1-> Some long words:
      \begin{itemize}
	\item<2-> \textbf{Distance} in spacetime is known as the \textbf{interval} -- in the same sense that the \textbf{point} is known as an \textbf{event}
	\item<3-> We did \textbf{trigonometric} rotations in \textbf{Euclidean space} -- the space we live in -- these \textbf{hyperbolic} rotations are in \textbf{Minkowskian spacetime}
	\item<4-> \textbf{Minkowskian spacetime} is flat, but has a \textbf{negative metric signature} -- tomorrow (Seminar 5) we will look at general relativity, in which the metric signature is the same, but because there is \textbf{gravity} the spacetime is \textbf{curved}
	\item<5-> You might want to look up some of these terms, but we don't have time (or the methematical development) to go into them
      \end{itemize}
  \end{itemize}
\end{frame}

\begin{frame}
  \frametitle{Lorentz rotations}
  \begin{itemize}
    \item<1-> Note to self: find some Lorentz transformations on Youtube
  \end{itemize}
\end{frame}

\begin{frame}
  \frametitle{Lorentz rotations}
  \begin{itemize}
    \item<1-> But I didn't go into why any of this \textbf{hyperbolic/Lorentz} rotation stuff is true\ldots
    \item<2-> In fact, we were happy with the simpler Galilean transformations before weren't we?
    \item<3-> Go back to $t$ and $x$ (i.e. \textbf{two dimensional case}) -- what if the \textbf{event} was a photon being \textbf{emitted} in the past at $t<0$ and some position $x$, and detected by us ($v=0$) at the \textbf{origin} of spacetime ($t_0=0$ and $x_0=0$) \textbf{Find the interval for the photon}:
      \begin{equation*}
	\sqrt{c^2t^2-x^2}
      \end{equation*}
  \end{itemize}
\end{frame}

\begin{frame}
  \frametitle{Lorentz rotations}
  \begin{itemize}
    \item<1-> We should find:
      \begin{equation*}
	\sqrt{c^2t^2-x^2}=0
      \end{equation*}
    \item<2-> This relies on the photon moving at $c$, right?
    \item<3-> And even if we start moving at $v$ we just spent ages proving that the \textbf{interval} of the photon's motion remains \textbf{unchanged}:
      \begin{equation*}
	\sqrt{c^2t'^2-x'^2}=0
      \end{equation*}
    \item<4-> So when we are moving at $v$, \textbf{what is the speed of the photon}?
    \item<5-> So all this \textbf{hyperbolic} rotation stuff just ensures that \textbf{the speed of light is the same, no matter how fast you are moving}
  \end{itemize}
\end{frame}

\begin{frame}
  \frametitle{Lorentz rotations}
  \begin{itemize}
    \item<1-> Finally, we \textbf{Lorentz} rotations:
      \begin{gather*}
	ct'=\frac{ct}{\sqrt{1-v^2/c^2}}-\frac{vx/c}{\sqrt{1-v^2/c^2}},\\
	x'=\frac{x}{\sqrt{1-v^2/c^2}}-\frac{vt}{\sqrt{1-v^2/c^2}},
      \end{gather*}
    \item<2-> And \textbf{Galilean} transformations:
      \begin{gather*}
	t'=t,\\
	x'=x-vt,
      \end{gather*}
    \item<3-> \textbf{Explain why Galilean transformations are okay so long as} $v\ll c$
  \end{itemize}
\end{frame}

\begin{frame}
  \frametitle{Lorentz rotations}
  \begin{itemize}
    \item<1-> We also have \textbf{relativistic} momentum:
      \begin{gather*}
	p=\frac{mv}{\sqrt{1-v^2/c^2}}
      \end{gather*}
    \item<2-> And \textbf{non-relativistic} momentum:
      \begin{gather*}
	p=mv
      \end{gather*}
    \item<3-> \textbf{Explain why they agree at} $v\ll c$
  \end{itemize}
\end{frame}

\begin{frame}
  \frametitle{Lorentz rotations}
  \begin{itemize}
    \item<1-> Finally we have \textbf{relativistic} energy:
      \begin{gather*}
	\mathcal{  E}=\frac{mc^2}{\sqrt{1-v^2/c^2}}
      \end{gather*}
    \item<2-> \textbf{What happens at} $v\ll c$?
    \item<3-> Should have \textbf{non-relativistic} energy:
      \begin{gather*}
	\mathcal{  E}=mc^2+\frac{1}{2}mv^2
      \end{gather*}
  \end{itemize}
\end{frame}

\begin{frame}
  \frametitle{Lorentz rotations}
  \begin{itemize}
    \item<1-> Should have \textbf{non-relativistic} energy:
      \begin{gather*}
	\mathcal{  E}=mc^2+\frac{1}{2}mv^2
      \end{gather*}
    \item<2-> What happens if the particle is standing still?
      \begin{equation*}
	\mathcal{  E}=mc^2
      \end{equation*}
    \item<3-> \textbf{The end.}
  \end{itemize}
\end{frame}

\begin{frame}
  \center
  \frametitle{Astrophysical phenomena}
  \includegraphics[height=5cm]{einstein.jpg}
\end{frame}

\end{document}


